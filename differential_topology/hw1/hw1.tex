\documentclass{article}
\usepackage[utf8]{inputenc}

\usepackage[letterpaper, margin=1in]{geometry}
\usepackage{amsmath} 
\usepackage{amsfonts} 
\usepackage{amssymb}
\usepackage{amsthm}
\usepackage{mathtools}
\usepackage{graphicx}
\usepackage{physics}
\usepackage{caption}
\usepackage{mdframed}
\usepackage{comment}
\usepackage{lipsum}
\usepackage{xcolor}
\usepackage[shortlabels]{enumitem}
\usepackage{titlesec}
\usepackage{tikz}

\usepackage{pdfpages}
\includepdfset{pages=-}

\usepackage{hyperref}
\hypersetup{
    colorlinks=true,
    linkcolor=red,
    filecolor=magenta,      
    urlcolor=blue,
}

\usepackage{fancyhdr}
    \pagestyle{fancy}
    \fancyhead[L]{Soleil Demick}
    \fancyhead[R]{MA 104 HW \#1}
    \fancyhead[C]{\thepage}
    \fancyfoot[]{}

\usepackage{setspace}
    % \doublespacing
    \setlength{\parskip}{1em}
    \setlength{\parindent}{0em}

\newcommand{\captioncenter}{\captionsetup{justification=centering,margin=2cm}}
\newcommand{\numberthis}{\addtocounter{equation}{1}\tag{\theequation}}

\newtheorem{theorem}{Theorem}
\newtheorem{claim}{Claim}
\newtheorem{proposition}{Proposition}
\newtheorem{lemma}{Lemma}
\newtheorem{definition}{Definition}
\renewcommand{\qedsymbol}{$\blacksquare$}

% Make every path printing in displaystyle, even inline
\everymath{\displaystyle}
\allowdisplaybreaks

% Line break in tabular/array cell
\newcommand{\specialcelltabular}[2][c]{%
  \begin{tabular}[#1]{@{}c@{}}#2\end{tabular}}
\newcommand{\specialcellarray}[2][c]{%
  \begin{array}[#1]{@{}c@{}}#2\end{array}}

% Macros
\renewcommand{\epsilon}{\varepsilon}
\newcommand{\problem}[1]{\textbf{Problem #1}.}
\newcommand{\todo}[1][]{[\textcolor{red}{TODO\ifstrempty{#1}{}{: #1}}]}
\newcommand{\inv}{^{-1}}
\renewcommand{\norm}[1]{\Vert{#1}\Vert}
\DeclareMathOperator{\id}{id}
\newcommand{\set}[1]{\left\{{#1}\right\}}
\newcommand{\suchthat}{\mid}
\newcommand{\forwardcase}{\(\boxed{\implies}\ \)}
\newcommand{\backwardcase}{\(\boxed{\impliedby}\ \)}

% Common symbols
%% Blackboard bold symbols
\newcommand{\bZ}{\mathbb{Z}}
\newcommand{\bN}{\mathbb{N}}
\newcommand{\bQ}{\mathbb{Q}}
\newcommand{\bR}{\mathbb{R}}
\newcommand{\bC}{\mathbb{C}}

\newcommand{\bS}{\mathbb{S}}
\newcommand{\bT}{\mathbb{T}}
\newcommand{\bP}{\mathbb{P}}

%% Assignment Specific Symbols
\newcommand{\onehalf}{\textstyle{\frac{1}{2}}}
\newcommand{\Oplus}{O_+}
\newcommand{\Ominus}{O_-}

% Vectors
\newcommand{\vN}{\vb{N}}
\newcommand{\vS}{\vb{S}}
\newcommand{\vp}{\vb{p}}
\newcommand{\vq}{\vb{q}}
\newcommand{\vL}{\vb{L}}

% Dots, Tildes, Hats


\begin{document}

\problem{1}
Consider  \(\bS^n\) as a subset of \(\bR^{n+1}\). Denote the north pole by \(\vN = (0,\dots,0,1)\) and the south pole by \(\vS = (0,\dots,0,-1)\). For any point \(\vp = (x_1, \dots, x_{n+1}) \in \bS^2\), we compute its north-pole and south-pole stereographic projections, \(\phi_N\) and \(\phi_S\) respectively, onto the equatorial hyperplane \(E\) of points \(\vq = (X_1, \dots, X_n, 0)\), and their associated inverses and transition functions.

We compute the north-pole projection \(\phi_N : \bS^2 \to E\). The line intersecting \(\vN\) and \(\vp\) is parameterized by
\begin{align*}
    \vL(t) = \vN + t(\vp - \vN) = (tx_1, \dots, tx_n, 1 + t(x_{n+1}-1)).
\end{align*}
Solving for when this line intersects \(E\):
\begin{align*}
    0 = 1 + t(x_{n+1}-1) \implies t = \frac{1}{1-x_{n+1}}.
\end{align*}
Hence
\begin{align*}
    \phi_N(\vp) = (X_1, \dots, X_n, 0) = \qty(\frac{x_1}{1-x_{n+1}}, \dots, \frac{x_n}{1-x_{n+1}}, 0).
\end{align*}
We now compute \(\phi_N\inv\). For \(\vq = (X_1, \dots, X_n, 0) \in E\), the line intersecting \(\vq\) and \(\vN\) is parameterized by
\begin{align*}
    \tilde{\vL}(t)
    & = \vq + t(\vN-\vq)
    \\
    & = (X_1 - tX_1, \dots, X_n - tX_n, t)
    \\
    & = ((1-t)X_1, \dots, (1-t)X_n, t).
\end{align*}
Solving for \(\norm{\tilde{\vL}} = 1\):
\begin{align*}
    1 
    & = (1-t)^2 X_1^2 + \cdots + (1-t)^2 X_n^2 + t^2
    \\
    & = (1-t)^2\norm{\vq}^2 + t^2
    \\ \implies
    0
    & = (1-t)^2\norm{\vq}^2 - (1-t^2)
    \\
    & = (1-t)((1-t)\norm{\vq}^2 - (1+t)).
\end{align*}
The solution \(t = 1\) gives \(\tilde{\vL} = \vN\). Hence, \(\vp\) occurs when
\begin{align*}
    t = \frac{\norm{\vq}^2 - 1}{\norm{\vq}^2 + 1}.
\end{align*}
Thus
\begin{align*}
    \phi_N\inv(\vq)
    & = (x_1, \dots, x_n, x_{n+1})
    \\
    & = \qty(\qty(1-\frac{\norm{\vq}^2 - 1}{\norm{\vq}^2 + 1})X_1, \dots, \qty(1-\frac{\norm{\vq}^2 - 1}{\norm{\vq}^2 + 1})X_n, \frac{\norm{\vq}^2 - 1}{\norm{\vq}^2 + 1})
    \\
    & = \qty(\frac{2}{\norm{\vq}^2+1}X_1, \dots, \frac{2}{\norm{\vq}^2+1} X_n, \frac{\norm{\vq}^2-1}{\norm{\vq}^2+1}).
\end{align*}

By symmetry, \(\phi_S\) and \(\phi_S\inv\) are given by the above formulas with \(x_{n+1} \mapsto -x_{n+1}\):
\begin{align*}
    & \phi_S(\vp) = \qty(\frac{x_1}{1+x_{n+1}}, \dots, \frac{x_n}{1+x_{n+1}}, 0)
    \\
    & \phi_S\inv(\vq) = \qty(\frac{2}{\norm{\vq}^2+1}X_1, \dots, \frac{2}{\norm{\vq}^2+1}X_n, -\frac{\norm{\vq}^2-1}{\norm{\vq}^2+1}).
\end{align*}

The transition functions are thus given by
\begin{align*}
    \phi_S \circ \phi_N\inv(\vq)
    & = \phi_S \qty(\frac{2}{\norm{\vq}^2+1}X_1, \dots, \frac{2}{\norm{\vq}^2+1} X_n, \frac{\norm{\vq}^2-1}{\norm{\vq}^2+1})
    \\
    & = \qty(\frac{\frac{2}{\norm{\vq}^2+1}X_1}{1 + \frac{\norm{\vq}^2-1}{\norm{\vq}^2+1}}, \dots, \frac{\frac{2}{\norm{\vq}^2+1}X_n}{1 + \frac{\norm{\vq}^2-1}{\norm{\vq}^2+1}}, 0)
    \\
    & = \qty(\frac{X_1}{\norm{\vq}^2}, \dots, \frac{X_n}{\norm{\vq}^2}, 0)
\end{align*}
and
\begin{align*}
    \phi_N \circ \phi_S\inv(\vq)
    & = \phi_N \qty(\frac{2}{\norm{\vq}^2+1}X_1, \dots, \frac{2}{\norm{\vq}^2+1}X_n, -\frac{\norm{\vq}^2-1}{\norm{\vq}^2+1})
    \\
    & = \qty(\frac{\frac{2}{\norm{\vq}^2+1}X_1}{1 + \frac{\norm{\vq}^2-1}{\norm{\vq}^2+1}}, \dots, \frac{\frac{2}{\norm{\vq}^2+1}X_n}{1 + \frac{\norm{\vq}^2-1}{\norm{\vq}^2+1}}, 0)
    \\
    & = \qty(\frac{X_1}{\norm{\vq}^2}, \dots, \frac{X_n}{\norm{\vq}^2}, 0).
\end{align*}


\problem{2}
Represent \(\bT^2\) by its fundamental polygon, fit to \([0, 1]^2 \subset \bR^2\). We choose for a chart the following open sets in \(\bT^2\):
\begin{align*}
    & U_1 = (0, 1)^2
    \\
    & U_2 = ([0, \onehalf) \cup (\onehalf, 1]) \times (0, 1)
    \\
    & U_3 = (0, 1) \times ([0, \onehalf) \cup (\onehalf, 1])
    \\
    & U_4 = ([0, \onehalf) \cup (\onehalf, 1])^2
\end{align*}
together with the respective homeomorphisms \(\phi_i : \bT^2 \to \bR^2\), \(i = 1,\dots,4\):
\begin{align*}
    \begin{array}{ll}
        \phi_1(x,y) = (x,y), & \phi_1\inv(x,y) = (x,y)
        \\
        \phi_2(x,y) = (x+\onehalf\mod{1}, y) & \phi_2\inv(x,y) = (x-\onehalf\mod{1}, y)
        \\
        \phi_3(x,y) = (x, y+\onehalf\mod{1}) & \phi_3\inv(x,y) = (x,y-\onehalf\mod{1})
        \\
        \phi_4(x,y) = (x+\onehalf\mod{1}, y+\onehalf\mod{1}) & \phi_4\inv(x,y) = (x-\onehalf\mod{1}, y-\onehalf\mod{1}).
    \end{array}
\end{align*}
We thus have the following transition maps:
\begin{align*}
    \begin{array}{ll}
        \phi_1\circ\phi_2\inv(x,y) = (x-\onehalf\mod{1}, y), & \phi_2\circ\phi_1\inv(x,y) = (x+\onehalf\mod{1}, y)
        \\
        \phi_1\circ\phi_3\inv(x,y) = (x,y-\onehalf\mod{1}), & \phi_3\circ\phi_1\inv(x,y) = (x, y+\onehalf\mod{1})
        \\
        \phi_1\circ\phi_4\inv(x,y) = (x-\onehalf\mod{1}, y-\onehalf\mod{1}), & \phi_4\circ\phi_1\inv(x,y) = (x+\onehalf\mod{1}, y+\onehalf\mod{1})
        \\
        \phi_2\circ\phi_3\inv(x,y) = (x+\onehalf\mod{1}, y-\onehalf\mod{1}), & \phi_3\circ\phi_2\inv(x,y) = (x-\onehalf\mod{1}, y+\onehalf\mod{1})
        \\
        \phi_2\circ\phi_4\inv(x,y) = (x,y-\onehalf\mod{1}), & \phi_4\circ\phi_2\inv(x,y) = (x,y+\onehalf\mod{1})
        \\
        \phi_3\circ\phi_4\inv(x,y) = (x-\onehalf\mod{1}, y), & \phi_4\circ\phi_3\inv(x,y) = (x+\onehalf\mod{1}, y).
    \end{array}
\end{align*}

\problem{3}
We have \(\bR\bP^n = \set{[x_0:\cdots:x_n]}\) with the atlas \(\set{(U_i, \phi_i)}_{i=0}^n\) given by 
\begin{align*}
    & U_i = \set{[x_0:\cdots:x_n] \suchthat x_i \neq 0}
    \\
    & \phi_i : U_i \to \bR^n,\ [x_0:\cdots:x_n] \mapsto \qty(\frac{x_0}{x_i}, \dots, \widehat{\frac{x_i}{x_i}}, \dots, \frac{x_n}{x_i}).
\end{align*}
We compute the inverse of the chart maps. For a fixed index \(1 \leq i \leq n\), we label our coordinates in \(\bR^n\) by \((X_0, \dots, \widehat{X_i}, \dots, X_n)\). Then
\begin{align*}
    \phi_i\inv(X_0, \dots, \widehat{X_i}, \dots, X_n) = [X_0 : \cdots : X_{i-1} : 1 : X_{i+1} : \cdots : X_n]
\end{align*}
because
\begin{align*}
    \phi_i([X_0 : \cdots : X_{i-1} : 1 : X_{i+1} : \cdots : X_n])
    & = \qty(\frac{X_0}{1}, \dots, \frac{X_{i-1}}{1}, \frac{X_{i+1}}{1}, \dots, \frac{X_n}{1}).
\end{align*}
Therefore, for \(i \neq j\), the transition function \(\phi_i \circ \phi_j\inv\) is given by
\begin{align*}
    \phi_i\circ\phi_j\inv(X_0, \dots, \widehat{X_j}, \dots, X_n)
    & = \phi_i([X_0 : \cdots : X_{j-1} : 1 : X_{j+1} : X_n])
    \\
    & = \qty(\frac{X_0}{X_i}, \dots, \widehat{\frac{X_i}{X_i}}, \dots, \frac{X_{j-1}}{X_i}, \frac{1}{X_i}, \frac{X_{j+1}}{X_i}, \dots, \frac{X_n}{X_i}).
\end{align*}

\problem{4} e choose for an atlas for \(\bC\bP^n\) one analogous to the atlas above for \(\bR\bP^n\). Consider \(\set{(U_i, \phi_i)}_{i=0}^n\) given by
\begin{align*}
    & U_i = \set{[z_0 : \cdots : z_n] \suchthat z_i \neq 0}
    \\
    & \phi_i : U_i \to \bC^n,\ [z_0 : \cdots : z_n] \mapsto \qty(\frac{z_0}{z_i}, \dots, \widehat{\frac{z_i}{z_i}}, \dots, \frac{z_n}{z_i}).
\end{align*}
The inverses and transition functions are exactly as above, substituting \(x_i \mapsto z_i\) and \(X_i \mapsto Z_i\). For \(i \neq j\), define the component functions \((f_1, \dots, \widehat{f_i}, \dots, f_n)\) of \(\phi_i \circ \phi_j\inv\) by
\begin{align*}
    & f_k(Z_0, \dots, \widehat{Z_j}, \dots, Z_n) = \frac{Z_k}{Z_i},\ k \neq i,\ k \neq j
    \\
    & f_j(Z-0, \dots, \widehat{Z_j}, \dots, Z_n) = \frac{1}{Z_i}.
\end{align*}

In the first case, \(f_k\) is independent of and thus holomorphic with respect to all \(Z_r\), where \(r \neq k\) and \(r \neq i\). Letting \(Z_k = X_k + IY_j\) and \(Z_i = X_i + IY_i\), where \(I\) is the imaginary unit, we get that
\begin{align*}
    f_k 
    & = \frac{X_k + IY_k}{X_i + IY_i}
    \\
    & = \frac{X_k X_i + Y_k Y_i}{X_i^2 + Y_i^2} + I \frac{Y_k X_i - X_k Y_i}{X_i^2 + Y_i^2}
    \\
    & := u_k + Iv_k.
\end{align*}
Then we have that
\begin{align*}
    \pdv{u_k}{X_k}
    & = \frac{X_i}{X_i^2 + Y_i^2}
    \\
    \pdv{v_k}{Y_k}
    & = \frac{X_i}{X_i^2 + Y_i^2}
    \\
    \pdv{u_k}{Y_k}
    & = \frac{Y_i}{X_i^2 + Y_i^2}
    \\
    \pdv{v_k}{X_k}
    & = -\frac{Y_i}{X_i^2 + Y_i^2},
\end{align*}
satisfying the Cauchy-Riemann equations in \(Z_k\), and 
\begin{align*}
    \pdv{u_k}{X_i}
    & = \frac{X_k(Y_i^2 - X_i^2)}{(X_i^2 + Y_i^2)^2}
    \\
    \pdv{v_k}{Y_i}
    & = -\frac{X_k(X_i^2 - Y_i^2)}{(X_i^2 + Y_i^2)^2} = \frac{X_k(Y_i^2 - X_i^2)}{(X_i^2 + Y_i^2)^2}
    \\
    \pdv{u_k}{Y_i}
    & = \frac{Y_k(X_i^2 - Y_i^2)}{(X_i^2 + Y_i^2)^2}
    \\
    \pdv{v_k}{X_i}
    & = \frac{Y_k(Y_i^2 - X_i^2)}{(X_i^2 + Y_i^2)^2} = -\frac{Y_k(X_i^2 - Y_i^2)}{(X_i^2 + Y_i^2)^2}.
\end{align*}
Hence \(f_k\) is holomorphic in all coordinates. In the second case, \(f_j\) is only dependent on \(Z_i\). We have
\begin{align*}
    f_j
    & = \frac{1}{X_i + IY_i}
    \\
    & = \frac{X_i}{X_i^2 + Y_i^2} - I\frac{Y_i}{X_i^2 + Y_i^2}
    \\
    & := u_j + Iv_j.
\end{align*}
We have that
\begin{align*}
    \pdv{u_j}{X_i}
    & = \frac{Y_i^2 - X_i^2}{(X_i^2 + Y_i^2)^2}
    \\
    \pdv{v_j}{Y_i}
    & = -\frac{X_i^2 - Y_i^2}{(X_i^2 + Y_i^2)^2} = \frac{Y_k(X_i^2 - Y_i^2)}{(X_i^2 + Y_i^2)^2}
    \\
    \pdv{u_j}{Y_i}
    & = -\frac{2X_iY_i}{(X_i^2 + Y_i^2)}
    \\
    \pdv{v_j}{X_i}
    & = \frac{2X_iY_i}{(X_i^2 + Y_i^2)^2}.
\end{align*}
Therefore, the transition functions are all holomorphic, and \(\bC\bP^n\) is a complex manifold.


\problem{5} Define \(\Oplus = (0, 1) \in X\) and \(\Ominus = (0, -1) \in X\). Let \(q : X \to M\) be the quotient map. For any \(p \in X \setminus \set{\Oplus, \Ominus}\) with \(x\)-coordinate \(p_x\), \(q(p) = [p] = \set{(p_x, 1), (p_x, -1)}\). Meanwhile, \(q(\Oplus) = [\Oplus] = \set{\Oplus}\) and \(q(\Ominus) = [\Ominus] = \set{\Ominus}\). 

We claim that \(M\) is locally Euclidean. For \([p] \in M\) with \(p \notin \set{\Oplus, \Ominus}\), define the neighborhood
\begin{align*}
    N_\epsilon([p]) := \set{[p'] \in M \suchthat \abs{p_x - p'_x} < \epsilon}, 
\end{align*}
where \(\epsilon < \abs{p_x}\). The preimage of this set under \(q\) is \((p_x-\epsilon, p_x+\epsilon) \times \set{1,-1}\), which is open in \(X \subset \bR^2\), thus \(N_\epsilon([p])\) is open. The neighborhood is homeomorphic to \((p_x-\epsilon, p_x+\epsilon)\) by the map \(\phi : [p] \mapsto p_x\). 

For \([\Oplus]\), define \(N_\epsilon([\Oplus])\) as above, with only that \(\epsilon > 0\). The preimage of this set is
\begin{align*}
    q\inv(N_\epsilon([\Oplus])) = [((-\epsilon, 0) \cup (0, \epsilon)) \times \set{1, -1}] \cup \set{\Oplus},
\end{align*}
which is open, hence \(N_\epsilon([\Oplus])\) is open. This neighborhood is homeomorphic to \((-\epsilon, \epsilon)\) by the map \(\phi\). An analogous construction holds for \(\Ominus\). Thus, \(M\) is locally Euclidean.

We now claim that \(M\) is second-countable. The rays \(\set{[p] \in M \suchthat p_x < 0 \text{ or } p_x > 0}\) have the usual countable basis of rational-radius open balls centered at rational points. We additionally add to this basis the collection of open balls \(N_\epsilon([\Oplus])\) and \(N_\epsilon([\Ominus])\) where \(\epsilon \in \bQ\). Hence \(M\) is second-countable.

Finally, we claim that \(M\) is not Haussdorff. For every open set \(U_1\) containing \([\Oplus]\) and \(U_2\) containing \([\Ominus]\), there exist neighborhoods \(N_{\epsilon_1}([\Oplus]) \subseteq U_1\) and \(N_{\epsilon_2}([\Ominus]) \subseteq U_2\) for some \(\epsilon_1, \epsilon_2 > 0\). Therefore, if \(\epsilon = \min\set{\epsilon_1, \epsilon_2}\), then we have the neighborhoods \(N_{\epsilon}([\Oplus]) \subseteq U_1\) and \(N_{\epsilon}([\Ominus]) \subseteq U_2\). But these neighborhoods have nonempty intersection \(N_{\epsilon/2}([-\epsilon/2]) \cup N_{\epsilon/2}([\epsilon/2])\), therefore \(M\) is not Haussdorff.


\problem{6}
Let \(M, N\) be smooth manifolds, and let \(F : M \to N\).

\forwardcase Suppose \(F\) satisfies Definition 1. Let \(p \in M\) be arbitrary. Since \(M\) is a manifold, there exists a chart \((U, \phi)\) with \(p \in U \subseteq M\). Similarly, since \(F(p) \in N\) and \(N\) is a manifold, there exists a chart \((V, \psi)\) with \(F(p) \in V \subseteq N\). By Definition 1, the map
\begin{align*}
    \psi \circ F \circ \phi\inv : \phi(U \cap F\inv(V)) \to \psi(V)
\end{align*}
is smooth. Hence \(F\) satisfies Definition 2.

\backwardcase Suppose \(F\) satisfies Definition 2 using the charts \((U, \phi)\) with \(p \in U \subseteq M\) and \((V, \psi)\) with \(F(p) \in V \subseteq N\). Consider any charts \((U', \phi')\) with \(p \in U' \subseteq M\) and \((V', \psi')\) with \(F(p) \in V' \subseteq N\). Then
\begin{align*}
    \psi' \circ F \circ (\phi')\inv
    & = (\psi' \circ \psi\inv) \circ (\psi \circ F \circ \phi\inv) \circ (\phi \circ (\phi')\inv) : \phi'(U' \cap F\inv(V')) \to \psi(V')
\end{align*}
is smooth, since the transition functions of smooth manifolds are smooth and the composition of smooth maps is smooth. Hence \(F\) satisfies Definition 1.

Therefore, the two definitions are equivalent.

\problem{7} See the following figures.

\includepdf{hw1-surfaces.pdf}


\end{document}